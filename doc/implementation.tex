%!TEX root = main.tex

This chapter describes the technical details of the implementation of signature-based partition refinement.
It contains a condensed version of several sections from~\cite{DBLP:conf/tacas/DijkP16}.


\section{Decision diagram algorithms in Sylvan}


In {symbolic model checking}, sets of states and transitions are represented by their characteristic function, rather than stored individually.
With states described by $N$ Boolean variables,
a set $S\subseteq \mathbb{B}^N$ can be represented by its characteristic function $f\colon \mathbb{B}^N\rightarrow \mathbb{B}$, where $S=\{s\mid f(s)\}$.
%
%
%
Binary decision diagrams (BDDs) are a concise and canonical representation of Boolean functions.

An (ordered) BDD is a directed acyclic graph with leaves $0$ and $1$. Each internal node has a variable label $x_i$ and two outgoing edges labeled $0$ and $1$. Variables are encountered along each path according to a fixed variable ordering. Duplicate nodes and nodes with two identical outgoing edges are forbidden.
%
%\begin{definition}
%An (ordered) BDD is a directed acyclic graph with the following properties:
%\begin{compactenum}
%\item There is a single root node and two terminal nodes 0 and 1.
%\item Each non-terminal node $p$ has a variable label $x_i$ and two outgoing edges, labeled 0 and 1; we write $\text{lvl}(p)=i$ and use $p[0]$ and $p[1]$ for the children.
%\item For each edge from node $p$ to non-terminal node $q$, $\text{lvl}(p)<\text{lvl}(q)$.
%%\item There are no {duplicate} nodes, i.e.,\\$\forall p\forall q\cdot(\text{lvl}(p)=\text{lvl}(q)\wedge p[0]=q[0]\wedge p[1]=q[1])\rightarrow p=q$.
%\item There are no duplicate nodes $\langle x_i,p[0],p[1]\rangle$ (i.e., maximum sharing)
%\item There are no redundant nodes (nodes with $p[0]=p[1]$)
%\end{compactenum}
%\end{definition}
%
%
%%
It is well known that for a fixed variable ordering, every Boolean function is represented by a unique BDD.


In addition to BDDs with leaves $0$ and $1$, multi-terminal binary decision diagrams have been proposed with leaves other than $0$ and $1$, representing functions from the Boolean space $\mathbb{B}^N$ onto any finite set. For example, MTBDDs can have leaves representing integers (encoding $\mathbb{B}^N\rightarrow\mathbb{N}$), floating-point numbers (encoding $\mathbb{B}^N\rightarrow\mathbb{R}$) and rational numbers (encoding $\mathbb{B}^N\rightarrow\mathbb{Q}$).
Partial functions are supported using a terminal leaf $\bot$.% (in practice we use the Boolean leaf $0$ for non-Boolean codomains).



Sylvan~\cite{DBLP:conf/tacas/DijkP15} implements parallelized operations on decision diagrams using parallel data structures and work-stealing.
Work-stealing is a load balancing method for task-based parallelism.
Recursive operations, such as most BDD operations, implicitly form a tree of tasks.
Independent subtasks are stored in queues and idle processors steal tasks from the queues of busy processors.


\begin{algorithm}[t!]
\Def{\Apply{$x$, $y$, $\textsf{F}$}}{
    \lIf{$(x,y,\textsf{F})\in\textsf{cache}$}{
        \Return{$\textsf{cache}[(x,y,\textsf{F})]$}\tcc*[f]{get from cache}    
    }
    \lIf{$x$ and $y$ are terminals}{
        \Return{$\textsf{F}(x,y)$}\tcc*[f]{apply operator \textsf{F}}
    }
    $v$ = topVar($x$,$y$) \;
    low $\leftarrow$ \Apply{$x_{v=0}$, $y_{v=0}$, $\textsf{F}$} \tcc*[r]{execute in parallel}
    high $\leftarrow$ \Apply{$x_{v=1}$, $y_{v=1}$, $\textsf{F}$} \;
    result $\leftarrow$ \BDDnode{$v$, low, high} \tcc*[r]{compute result}
    $\textsf{cache}[(x,y,\textsf{F})]$ $\leftarrow$ result \tcc*[r]{put in cache}
    \Return{result}
}
\caption{Generic algorithm that applies a binary operator \textsf{F} to BDDs $x$ and $y$.}
\label{alg:generic}
\end{algorithm}



Algorithm~\ref{alg:generic} describes the implementation of a generic binary operation $\textsf{F}$.
%
%
%
BDD operations mainly consist of consulting an operation cache, performing some recursive step, and creating new BDD nodes using a unique table.
The operation cache is required to reduce the time complexity of BDD operations from exponential to polynomial in the size of the BDDs.
Sylvan uses a single shared unique table for all BDD nodes and a single shared operation cache for all operations.
To obtain high performance in a multi-core environment, the datastructures for the BDD nodes and the operation cache must be highly scalable.
Sylvan implements several non-blocking datastructures to enable good speedups.



To compute symbolic signature-based partition refinement, several basic operations must be supported by the BDD package (see also \cite{DBLP:conf/atva/WimmerHHSB06}).
%
%
%
Sylvan implements basic operations such as $\wedge$ and \texttt{if-then-else}, and existential quantification $\exists$.
Negation $\neg$ is performed in constant time using complement edges.
To compute relational products of transition systems, there are operations \texttt{relnext} (to compute successors) and \texttt{relprev} (to compute predecessors and to concatenate relations), which combine the relational product with variable renaming.
The operation \texttt{and\_exists} computes the traditional relational product without variable renaming.
Similar operations are also implemented for MTBDDs.
%
%
%
Sylvan is designed to support custom BDD algorithms.
We present two custom algorithms below.
%We also use custom extensions of traditional BDD algorithms to avoid variable renaming.



\section{Encoding of signature refinement}

We implement symbolic signature refinement similar to~\cite{DBLP:conf/atva/WimmerHHSB06}.
%
%
Unlike~\cite{DBLP:conf/atva/WimmerHHSB06}, we do not refine the partition with respect to a single block, but with respect to all blocks simultaneously.
% like in~\cite{DBLP:journals/entcs/BlomO03,DBLP:conf/qest/Derisavi07}.
%
%
%
We use a binary encoding with variables $s$ for the current state, $s'$ for the next state, $a$ for the action labels and $b$ for the blocks.
We order BDD variables $a$ and $b$ after $s$ and $s'$, since this is required
to efficiently replace signatures $(a,b)$ by new block numbers $b$ (see below).
Variables $s$ and $s'$ are interleaved, which is common in the context of transition systems.



\begin{figure}[tp]
\begin{center}
\scalebox{0.9}{
\begin{tikzpicture}
\coordinate (a) at (250:2);
\coordinate (b) at (290:2);
\coordinate (c) at (250:3);
\coordinate (d) at (290:3);
\draw (0,0) -- (c);
\draw (0,0) -- (d);
\draw (a) -- (b);
\draw (c) -- (d);
\draw (0,-1.5) node {$s,s'$};
\draw (0,-2.35) node {$a$};
\draw (0,-4.5) node {BDD $\mathcal{T}(s,s',a)$};

\coordinate (a0) at (3,0);
\coordinate (aa) at ($(a0) + (250:2)$);
\coordinate (ab) at ($(a0) + (290:2)$);
\coordinate (ac) at ($(a0) + (250:3)$);
\coordinate (ad) at ($(a0) + (290:3)$);
\coordinate (ae) at ($(a0) + (250:4)$);
\coordinate (af) at ($(a0) + (290:4)$);
\draw (a0) -- (ae);
\draw (a0) -- (af);
\draw (aa) -- (ab);
%\draw (ac) -- (ae);
\%draw (ad) -- (af);
\draw (ac) -- (ad);
\draw (ae) -- (af);
\draw (3,-1.5) node {$s$};
\draw (3,-2.35) node {$a$};
\draw (3,-3.25) node {$b$};
\draw (3,-4.5) node {BDD $\sigma_T(s,a,b)$};

\coordinate (b0) at (6.5,0);
\coordinate (ba) at ($(b0) + (250:2)$);
\coordinate (bb) at ($(b0) + (290:2)$);
\coordinate (bc) at ($(b0) + (250:3)$);
\coordinate (bd) at ($(b0) + (290:3)$);
\coordinate (be) at ($(b0) + (250:4)$);
\coordinate (bf) at ($(b0) + (290:4)$);
\draw (b0) -- (ba);
\draw (b0) -- (bb);
\draw (ba) -- (bb);
\draw (bc) -- (be);
\draw (bd) -- (bf);
\draw (bc) -- (bd);
\draw (be) -- (bf);
\draw ($(b0) + (0,-1.5)$) node {$s$};
\draw ($(b0) + (0,-3.25)$) node {$b$};
\draw ($(b0) + (0,-4.5)$) node {MTBDD $\sigma_R(s,b)$};

\coordinate (c0) at (-3,0);
\coordinate (ca) at ($(c0) + (250:2)$);
\coordinate (cb) at ($(c0) + (290:2)$);
\coordinate (cc) at ($(c0) + (250:3)$);
\coordinate (cd) at ($(c0) + (290:3)$);
\coordinate (ce) at ($(c0) + (250:4)$);
\coordinate (cf) at ($(c0) + (290:4)$);
\draw (c0) -- (ca);
\draw (c0) -- (cb);
\draw (ca) -- (cb);
\draw (cc) -- (ce);
\draw (cd) -- (cf);
\draw (cc) -- (cd);
\draw (ce) -- (cf);
\draw ($(c0) + (0,-1.5)$) node {$s$};
\draw ($(c0) + (0,-3.25)$) node {$b$};
\draw ($(c0) + (0,-4.5)$) node {BDD $\mathcal{P}(s,b)$};
\end{tikzpicture}
}
\end{center}
\caption{Schematic overview of the BDDs in signature refinement}
\label{fig:bdds}
\end{figure}



To perform symbolic bisimulation we represent a number of sets by their characteristic functions. See also Figure~\ref{fig:bdds}.
\begin{itemize}
\item A set of states is represented by a BDD $\mathcal{S}(s)$;
% with $\mathcal{S}(s)=1\Leftrightarrow s\in S$
\item Transitions are represented by a BDD $\mathcal{T}(s,s',a)$;
\item Markovian transitions are represented by an MTBDD $\mathcal{R}(s,s')$, with leaves containing rational numbers ($\mathbb{Q}$);
\item Signatures $\textbf{T}$ and $\textbf{B}$ are represented by a BDD $\sigma_T(s,a,b)$;
\item Signatures $\textbf{R}^s$ and $\textbf{R}^b$ are represented by an MTBDD $\sigma_R(s,b)$.
\end{itemize}




In the literature,
%~\cite{DBLP:conf/cav/BoualiS92,DBLP:conf/tacas/Derisavi07,DBLP:conf/qest/Derisavi07,DBLP:conf/atva/WimmerHHSB06}, 
three methods have been proposed to represent $\pi$.
%
%
%
\begin{enumerate}
\item As an equivalence relation, using a BDD $\mathcal{E}(s,s')=1$ iff $s\equiv_{\pi} s'$.
\item As a partition, by assigning each block a unique number, encoded with variables $b$, using a BDD $\mathcal{P}(s,b)=1$ iff $s\in C_b$.
\item Using $k=\ceil{\log_2 n}$ BDDs $\mathcal{P}_{0},\dotsc,\mathcal{P}_{k-1}$ such that $\mathcal{P}_i(s)=1$ iff $s\in C_b$ and the $i^\text{th}$ bit of $b$ is 1. This requires significant time to restore blocks for the refinement procedure, but can require less memory.
\end{enumerate}

%We use variables $t$ and $b$ instead of $s$ and $b$ in $\mathcal{P}$ to avoid creating a custom BDD operation for computing $\exists t\colon \mathcal{T}(s,t,a) \wedge \mathcal{P}(t,b)$.
%Also, \cite{DBLP:conf/atva/WimmerHHSB06} often requires renaming operations to get $\mathcal{P}(t,b)$ from $\mathcal{P}(s,b)$, which is unnecessary and expensive.

%Furthermore, the encoding of block numbers is such that the the lowest significant bit comes first...

%We represent the partition by the assignment to a block number.
%An alternative method to represent the partition, used by \cite{DBLP:conf/cav/BoualiS92} and \cite{DBLP:journals/ijfcs/MummeC13}, stores the equivalence relation in a BDD $\mathcal{R}(s,t)$.
We choose to use method 2, since in practice the BDD of $\mathcal{P}(s,b)$ is smaller than the BDD of $\mathcal{E}(s,s')$.
Using $\mathcal{P}(s,b)$ also has the advantage of straight-forward signature computation.
%In cases where only few blocks contain multiple states, $\mathcal{P}(s,b)$ blows up to a large tree, assigning each state to a different block.
%In such cases, the BDD for $\mathcal{E}(s,t)$ would be much smaller, since less states are equivalent.
%
%
The logarithmic representation is incompatible with our approach, since we refine all blocks simultaneously. Their approach involves restoring individual blocks to the $\mathcal{P}(s,b)$ representation, performing a refinement step, and compacting the result to the logarithmic representation. Restoring all blocks simply computes the full $\mathcal{P}(s,b)$.

We represent Markovian transitions using rational numbers, since they offer better precision than floating-point numbers.
The manipulation of floating-point numbers typically introduces tiny rounding errors, resulting in different results of similar computations.
This significantly affects bisimulation reduction, often resulting in finer partitions than the maximal bisimulation~\cite{DBLP:conf/mmb/WimmerB10}.


\section{The \texttt{refine} algorithm}
\label{sec:refine}
%!TEX root = main.tex

Partition refinement consists of two steps: computing the signatures and computing the next partition.
Given the signatures $\sigma_T$ and/or $\sigma_R$ for the current partition $\pi$,
the new partition can be computed as follows.



Since the chosen variable ordering has variables $s,s'$ before $a,b$,
each path in $\sigma$
ends in a (MT)BDD representing the signature for the states encoded by that path.
%, or
%terminal node 0 for the empty signature.
%
%
For $\sigma_T$, every path that assigns values to $s$ ends in a BDD on $a,b$.
%
%
For $\sigma_R$, every path that assigns values to $s$ ends in a MTBDD on $b$ with rational leaves.
%




Wimmer et al.~\cite{DBLP:conf/atva/WimmerHHSB06} present a BDD operation \texttt{refine} that ``replaces''  these sub-(MT)BDDs by the BDD representing a unique block number for each distinct signature.
The result is the BDD of the next partition.
%
%
%
They use a global counter and a hash table to associate each signature with a unique block number.
%
%
This algorithm has the disadvantage that block number assignments are unstable.
%
%
There is no guarantee that a stable block has the same block number in the next iteration.
%
%
This has implications for the computation of the new signatures.
%
%
When the block number of a stable block changes, cached results of signature computation in earlier iterations cannot be reused.


\begin{algorithm}[t!]
\Def{\Refine{$\sigma$, $\mathcal{P}$}}{
    \lIf{$(\sigma,\mathcal{P},\textsf{iter})\in\textsf{cache}$}{
        \Return{$\textsf{cache}[(\sigma,\mathcal{P},\textsf{iter})]$}
    }
    $v$ = topVar($\sigma$, $\mathcal{P}$) \;
    \If{$v$ equals $s_i$ for some $i$}{
        \tcp{match paths on $s$ in $\sigma$ and $\mathcal{P}$}
        low $\leftarrow$ \Refine{$\sigma_{s_i=0}$, $\mathcal{P}_{s_i=0}$} \;
        high $\leftarrow$ \Refine{$\sigma_{s_i=1}$, $\mathcal{P}_{s_i=1}$} \;
        result $\leftarrow$ \BDDnode{$s_i$, low, high}
    }
    \Else{
        \tcp{$\sigma$ now encodes the state signature}
        \tcp{$\mathcal{P}$ now encodes the previous block}
%        \tcp{decode previous block number}
    		$B \leftarrow \texttt{decodeBlock}(\mathcal{P})$ \;
    		\tcp{try to claim block B if still free}
		\lIf{$\texttt{blocks}[B].\texttt{sig}=\bot$}{$\texttt{cas}(\texttt{blocks}[B].\texttt{sig},\bot,\sigma)$}
%		\tcp{check if B now contains $\sigma$}
		\lIf{$\texttt{blocks}[B].\texttt{sig}=\sigma$}{result $\leftarrow$ $\mathcal{P}$}
		\Else{
%		    \tcp{find a new block number and encode it}
		    $B \leftarrow \texttt{search\_or\_insert}(\sigma, B)$ \;
	        result $\leftarrow$ \texttt{encodeBlock}($B$)
        }
    }
    $\textsf{cache}[(\sigma,\mathcal{P},\textsf{iter})]$ $\leftarrow$ result \;
    \Return{result}
}
\caption{\texttt{refine}, the (MT)BDD operation that assigns block numbers to signatures, given a signature $\sigma$ and the previous partition $\mathcal{P}$.}
\label{alg:refine}
\end{algorithm}



We modify the \texttt{refine} algorithm to use the current partition to reuse the previous block number of each state.
%
%accept an extra parameter $\mathcal{P}(s,b)$ to compute the previous block number of each state.
%
This also allows refining a partition with respect to only a part of the signature, as described in Section~\ref{sec:sigref}.
%
%
The modification is applied such that it can be parallelized in Sylvan.
%
%
%Thus, a new partition is computed based on the signature and the block number in the previous partition.
%The result is the new partition on variables $s$ and $b$.
See Algorithm~\ref{alg:refine}.


The algorithm has two input parameters: the (MT)BDD $\sigma$ which encodes the (partial) signature for the current partition, and the BDD $\mathcal{P}$ which encodes the current partition.
%
%
%The result is a BDD representing the next partition.
%
%
The algorithm uses a global counter $\textsf{iter}$, which is the current iteration of partition refinement.
%
%
This is necessary since the cached results of the previous iteration cannot be reused.
%
%
It also uses and updates an array \texttt{blocks}, which contains the signature of each block in the new partition.
%
This array is cleared between iterations of partition refinement.


The implementation is similar to other BDD operations, featuring the use of the operation cache (lines~2 and~15) and a recursion step for variables in $s$ (lines~3--7), with the two recursive operations executed in parallel.
%
\texttt{refine} simultaneously descends in $\sigma$ and $\mathcal{P}$ (lines~5--6), matching the valuation of $s_i$ in $\sigma$ and $\mathcal{P}$.
%
%
Block assignment happens at lines~9--14.
%
We rely on the well-known atomic operation \texttt{compare\_and\_swap} (\texttt{cas}), which atomically compares and modifies a value in memory.
%
This is necessary so the algorithm is still correct when parallelized.
%
%
We use \texttt{cas} to claim a block number for the signature (line 10).
%
%
If the block number is already used for a different signature, then this block is being refined and we call a method \texttt{search\_or\_insert} to assign a new block number.
%
%
%A new block number is assigned only when the previous block number is already used for a different signature (thus the block is being split), by calling a method \texttt{search\_or\_insert}.
%
%
%
%The result is that previous block numbers are maximally reused; \texttt{search\_or\_insert} is only called when a block is split and the the previous block number has been reused already.

Different implementations of \texttt{search\_and\_insert} are possible.
%
We implemented a parallel hash table that uses a global counter for the next block number when inserting a new pair $(\sigma, B)$, similar to~\cite{DBLP:conf/atva/WimmerHHSB06}.
%
%
An alternative implementation that performed better in our experiments integrates the \texttt{blocks} array with a skip list.
%
%
A skip list is a probabilistic multi-level ordered linked list.
See~\cite{DBLP:journals/cacm/Pugh90}.

Our implementation of the skip list is restricted to at most 5 levels and supports only the \texttt{insert} operation.
%
%
We use a short-lived local lock at the lowest level to insert buckets,
and lock-free insertions using \texttt{cas} at higher levels.
%
%
%
Furthermore, by restricting the number of blocks to at most $2^{31}$, we only need 32 bytes for each bucket in the skip list:
%
%
$$\texttt{struct \{ uint64\_t sig; uint32\_t prev\_block; uint32\_t next[5]; \}}$$
%
Each bucket in the skip list contains the pair $(\sigma,B)$ (\texttt{sig} and \texttt{prev\_block}) and the 31-bit indices of the next bucket at each level.
The highest bit of \texttt{next[0]} is used as a lock, which is released when setting \texttt{next[0]} to a new value.


We implement \texttt{search\_or\_insert} as follows.
%
%
We traverse the skip list until either a bucket with $(\sigma,B)$ is found (and returned), or the bucket $B'$ that would immediately precede a bucket with $(\sigma,B)$.
%
%
We use \texttt{cas} to set the highest bit of \texttt{next[0]} and lock bucket $B'$.
%, promising to insert a bucket after $B''$.
%
%
This ensures that no other thread can insert $(\sigma,B)$ (or any other pair directly preceded by bucket $B'$) simultaneously.
%
%
A new block number $B''$ is generated using a global counter \texttt{next\_block}, which
is increased atomically with \texttt{cas}.
%
%
Bucket $B''$ is initialized and inserted into the skip list by updating \texttt{next[0]} of $B'$ with $B''$, which also releases the lock on $B'$.
%
%
Finally, the new bucket is inserted at a random number of higher levels using \texttt{cas}.

\section{Computing inert transitions}
\label{sec:inert}
%!TEX root = main.tex


\begin{algorithm}[tbp]
\Def{\Inert{$\mathcal{T}$, $\mathcal{P}^s$, $\mathcal{P}^{s'}$}}{
    \lIf{$(\mathcal{T},\mathcal{P}^s,\mathcal{P}^{s'})\in\textsf{cache}$}{
        \Return{$\textsf{cache}[(\mathcal{T},\mathcal{P}^s,\mathcal{P}^{s'})]$}
    }
    \tcp{find highest variable, interpreting $s_i$ in $\mathcal{P}^{s'}$ as $s'_i$}
    $v$ = topVar($\mathcal{T}$, $\mathcal{P}^s$, $\mathcal{P}^{s'}$) \;
    \If{$v$ equals $s_i$ for some $i$}{
        \tcp{match $s_i$ in $\mathcal{T}$ with $\mathcal{P}^s$}
        low $\leftarrow$ \Inert{$\mathcal{T}_{s_i=0}$, $\mathcal{P}^s_{s_i=0}$, $\mathcal{P}^{s'}$} \;
        high $\leftarrow$ \Inert{$\mathcal{T}_{s_i=1}$, $\mathcal{P}^s_{s_i=1}$, $\mathcal{P}^{s'}$} \;
        result $\leftarrow$ \BDDnode{$s_i$, low, high}
    }
    \ElseIf{$v$ equals $s'_i$ for some $i$}{
        \tcp{match $s'_i$ in $\mathcal{T}$ with $s_i$ in $\mathcal{P}^{s'}$}
        low $\leftarrow$ \Inert{$\mathcal{T}_{s'_i=0}$, $\mathcal{P}^s$, $\mathcal{P}^{s'}_{s_i=0}$} \;
        high $\leftarrow$ \Inert{$\mathcal{T}_{s'_i=1}$, $\mathcal{P}^s$, $\mathcal{P}^{s'}_{s_i=1}$} \;
        result $\leftarrow$ \BDDnode{$s'_i$, low, high}
    }
    \Else{
        \tcp{match the blocks $\mathcal{P}^s$ and $\mathcal{P}^{s'}$}
        \lIf{$\mathcal{P}^s\neq\mathcal{P}^{s'}$}{
            result $\leftarrow$ \texttt{False}
        }
        \lElse{
            result $\leftarrow$ $\mathcal{T}$
        }
    }
    $\textsf{cache}[(\mathcal{T},\mathcal{P}^s,\mathcal{P}^{s'}]$ $\leftarrow$ result \;
    \Return{result}
}
\caption{Computes the inert transitions of a transition relation $\mathcal{T}$ according to the block assignments to current states ($\mathcal{P}^s$) and next states ($\mathcal{P}^{s'}$).}
\label{alg:inert}
\end{algorithm}




\begin{figure}[tbp]
\begin{center}
\scalebox{0.9}{
\begin{tikzpicture}
\coordinate (a) at (250:2);
\coordinate (b) at (290:2);
\coordinate (c) at (250:3);
\coordinate (d) at (290:3);
\draw (0,0) -- (a);
\draw (0,0) -- (b);
\draw (a) -- (b);
\draw[dashed] (a) -- (c);
\draw[dashed] (b) -- (d);
\draw (0,-1.5) node {$s,s'$};

\coordinate (a0) at (3,0);
\coordinate (aa) at ($(a0) + (250:2)$);
\coordinate (ab) at ($(a0) + (290:2)$);
\coordinate (ac) at ($(a0) + (250:3)$);
\coordinate (ad) at ($(a0) + (290:3)$);
\coordinate (ae) at ($(a0) + (250:4)$);
\coordinate (af) at ($(a0) + (290:4)$);
\draw (a0) -- (ac);
\draw (a0) -- (ad);
\draw (aa) -- (ab);
\draw (ac) -- (ad);
\draw (3,-1.5) node {$s$};
\draw (3,-2.35) node {$b$};

\coordinate (b0) at (6,0);
\coordinate (ba) at ($(b0) + (250:2)$);
\coordinate (bb) at ($(b0) + (290:2)$);
\coordinate (bc) at ($(b0) + (250:3)$);
\coordinate (bd) at ($(b0) + (290:3)$);
\draw (b0) -- (ba);
\draw (b0) -- (bb);
\draw (ba) -- (bb);
\draw[dashed] (ba) -- (bc);
\draw[dashed] (bb) -- (bd);
\draw (6,-1.5) node {$s,s'$};

\coordinate (c0) at (-3,0);
\coordinate (ca) at ($(c0) + (250:2)$);
\coordinate (cb) at ($(c0) + (290:2)$);
\coordinate (cc) at ($(c0) + (250:3)$);
\coordinate (cd) at ($(c0) + (290:3)$);
\coordinate (ce) at ($(c0) + (250:4)$);
\coordinate (cf) at ($(c0) + (290:4)$);
\draw (c0) -- (cc);
\draw (c0) -- (cd);
\draw (ca) -- (cb);
\draw (cc) -- (cd);
\draw ($(c0) + (0,-1.5)$) node {$s$};
\draw ($(c0) + (0,-2.35)$) node {$b$};

\draw[densely dotted] ($(0,0)!0.50!(b)$) to[bend left=10] node[above] {match $s'=s$} ($(a0)!0.50!(aa)$);
\draw[densely dotted] ($(c0)!0.50!(cb)$) to[bend left=10] node[above] {match $s=s$} ($(0,0)!0.50!(a)$);
\draw[densely dotted] ($(cb)!0.5!(cd)$) to[bend right=10] node[below] {same block} ($(aa)!0.5!(ac)$);

\draw (0,-3.5) node {$\mathcal{T}$};
%\draw (1.5,-4.5) node {\&};
\draw (3,-3.5) node {$\mathcal{P}^{s'}$};
%\draw (-1.5,-4.5) node {\&};
\draw (-3,-3.5) node {$\mathcal{P}^s$};
%\draw (5,-4.5) node {$\Rightarrow$};
\draw (6,-3.5) node {$\textsf{inert}$};

\end{tikzpicture}
}
\end{center}
\caption{Schematic overview of the BDDs in the \texttt{inert} algorithm}
\label{fig:inert}
\end{figure}




To compute the set of inert $\tau$-transitions for branching bisimulation,
i.e., $s\smash\taupi s'$, 
or more generally,
to compute any inert transition relation $\rightarrow\!\cap\!\equiv$ where $\equiv$ is the equivalence relation corresponding to $\pi$ computed by $\mathcal{E}(s,s')=\exists b\colon \mathcal{P}(s,b) \wedge \mathcal{P}(s',b)$,
the expression $\mathcal{T}(s,s') \wedge \exists b\colon 
\mathcal{P}(s,b) \wedge \mathcal{P}(s',b)$ must be computed.
%
%
%
We compute $\rightarrow\!\cap\!\equiv$ directly using a custom BDD algorithm.
%
%
%
%
The \texttt{inert} algorithm takes parameters $\mathcal{T}(s,s')$ ($\mathcal{T}$ may contain other variables ordered after $s,s'$) and two copies of $\mathcal{P}(s,b)$: $\mathcal{P}^s$ and $\mathcal{P}^{s'}$.
%
%
%It optionally also takes a parameter $\mathcal{A}(a)$, representing a set of action labels to restrict the result to (e.g. $\tau$ for only inert $\tau$-transitions).
%
%
The algorithm matches $\mathcal{T}$ and $\mathcal{P}^s$ on valuations of variables $s$, and $\mathcal{T}$ and $\mathcal{P}^{s'}$ on valuations of variables $s'$.
%
%
See Algorithm~\ref{alg:inert}, and also Figure~\ref{fig:inert} for a schematic overview.
%
%
When in the recursive call all valuations to $s$ and $s'$ have been matched, with $S_s,S_{s'}\subseteq S$ the sets of states represented by these valuations, then $\mathcal{T}$ is the set of actions that label the transitions between states in $S_s$ and $S_{s'}$, $\mathcal{P}^s$ is the block that contains all $S_s$ and $\mathcal{P}^{s'}$ is the block that contains all $S_{s'}$.
Then if $\mathcal{P}^s\neq\mathcal{P}^{s'}$, the transitions are not inert and \texttt{inert} returns \texttt{False}, removing the transition from $\mathcal{T}$.
%
%
Otherwise, $\mathcal{T}$ (which may still contain other variables ordered after $s,s'$, such as action labels), is returned.



