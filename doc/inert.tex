%!TEX root = main.tex


\begin{algorithm}[tbp]
\Def{\Inert{$\mathcal{T}$, $\mathcal{P}^s$, $\mathcal{P}^{s'}$}}{
    \lIf{$(\mathcal{T},\mathcal{P}^s,\mathcal{P}^{s'})\in\textsf{cache}$}{
        \Return{$\textsf{cache}[(\mathcal{T},\mathcal{P}^s,\mathcal{P}^{s'})]$}
    }
    \tcp{find highest variable, interpreting $s_i$ in $\mathcal{P}^{s'}$ as $s'_i$}
    $v$ = topVar($\mathcal{T}$, $\mathcal{P}^s$, $\mathcal{P}^{s'}$) \;
    \If{$v$ equals $s_i$ for some $i$}{
        \tcp{match $s_i$ in $\mathcal{T}$ with $\mathcal{P}^s$}
        low $\leftarrow$ \Inert{$\mathcal{T}_{s_i=0}$, $\mathcal{P}^s_{s_i=0}$, $\mathcal{P}^{s'}$} \;
        high $\leftarrow$ \Inert{$\mathcal{T}_{s_i=1}$, $\mathcal{P}^s_{s_i=1}$, $\mathcal{P}^{s'}$} \;
        result $\leftarrow$ \BDDnode{$s_i$, low, high}
    }
    \ElseIf{$v$ equals $s'_i$ for some $i$}{
        \tcp{match $s'_i$ in $\mathcal{T}$ with $s_i$ in $\mathcal{P}^{s'}$}
        low $\leftarrow$ \Inert{$\mathcal{T}_{s'_i=0}$, $\mathcal{P}^s$, $\mathcal{P}^{s'}_{s_i=0}$} \;
        high $\leftarrow$ \Inert{$\mathcal{T}_{s'_i=1}$, $\mathcal{P}^s$, $\mathcal{P}^{s'}_{s_i=1}$} \;
        result $\leftarrow$ \BDDnode{$s'_i$, low, high}
    }
    \Else{
        \tcp{match the blocks $\mathcal{P}^s$ and $\mathcal{P}^{s'}$}
        \lIf{$\mathcal{P}^s\neq\mathcal{P}^{s'}$}{
            result $\leftarrow$ \texttt{False}
        }
        \lElse{
            result $\leftarrow$ $\mathcal{T}$
        }
    }
    $\textsf{cache}[(\mathcal{T},\mathcal{P}^s,\mathcal{P}^{s'}]$ $\leftarrow$ result \;
    \Return{result}
}
\caption{Computes the inert transitions of a transition relation $\mathcal{T}$ according to the block assignments to current states ($\mathcal{P}^s$) and next states ($\mathcal{P}^{s'}$).}
\label{alg:inert}
\end{algorithm}




\begin{figure}[tbp]
\begin{center}
\scalebox{0.9}{
\begin{tikzpicture}
\coordinate (a) at (250:2);
\coordinate (b) at (290:2);
\coordinate (c) at (250:3);
\coordinate (d) at (290:3);
\draw (0,0) -- (a);
\draw (0,0) -- (b);
\draw (a) -- (b);
\draw[dashed] (a) -- (c);
\draw[dashed] (b) -- (d);
\draw (0,-1.5) node {$s,s'$};

\coordinate (a0) at (3,0);
\coordinate (aa) at ($(a0) + (250:2)$);
\coordinate (ab) at ($(a0) + (290:2)$);
\coordinate (ac) at ($(a0) + (250:3)$);
\coordinate (ad) at ($(a0) + (290:3)$);
\coordinate (ae) at ($(a0) + (250:4)$);
\coordinate (af) at ($(a0) + (290:4)$);
\draw (a0) -- (ac);
\draw (a0) -- (ad);
\draw (aa) -- (ab);
\draw (ac) -- (ad);
\draw (3,-1.5) node {$s$};
\draw (3,-2.35) node {$b$};

\coordinate (b0) at (6,0);
\coordinate (ba) at ($(b0) + (250:2)$);
\coordinate (bb) at ($(b0) + (290:2)$);
\coordinate (bc) at ($(b0) + (250:3)$);
\coordinate (bd) at ($(b0) + (290:3)$);
\draw (b0) -- (ba);
\draw (b0) -- (bb);
\draw (ba) -- (bb);
\draw[dashed] (ba) -- (bc);
\draw[dashed] (bb) -- (bd);
\draw (6,-1.5) node {$s,s'$};

\coordinate (c0) at (-3,0);
\coordinate (ca) at ($(c0) + (250:2)$);
\coordinate (cb) at ($(c0) + (290:2)$);
\coordinate (cc) at ($(c0) + (250:3)$);
\coordinate (cd) at ($(c0) + (290:3)$);
\coordinate (ce) at ($(c0) + (250:4)$);
\coordinate (cf) at ($(c0) + (290:4)$);
\draw (c0) -- (cc);
\draw (c0) -- (cd);
\draw (ca) -- (cb);
\draw (cc) -- (cd);
\draw ($(c0) + (0,-1.5)$) node {$s$};
\draw ($(c0) + (0,-2.35)$) node {$b$};

\draw[densely dotted] ($(0,0)!0.50!(b)$) to[bend left=10] node[above] {match $s'=s$} ($(a0)!0.50!(aa)$);
\draw[densely dotted] ($(c0)!0.50!(cb)$) to[bend left=10] node[above] {match $s=s$} ($(0,0)!0.50!(a)$);
\draw[densely dotted] ($(cb)!0.5!(cd)$) to[bend right=10] node[below] {same block} ($(aa)!0.5!(ac)$);

\draw (0,-3.5) node {$\mathcal{T}$};
%\draw (1.5,-4.5) node {\&};
\draw (3,-3.5) node {$\mathcal{P}^{s'}$};
%\draw (-1.5,-4.5) node {\&};
\draw (-3,-3.5) node {$\mathcal{P}^s$};
%\draw (5,-4.5) node {$\Rightarrow$};
\draw (6,-3.5) node {$\textsf{inert}$};

\end{tikzpicture}
}
\end{center}
\caption{Schematic overview of the BDDs in the \texttt{inert} algorithm}
\label{fig:inert}
\end{figure}




To compute the set of inert $\tau$-transitions for branching bisimulation,
i.e., $s\smash\taupi s'$, 
or more generally,
to compute any inert transition relation $\rightarrow\!\cap\!\equiv$ where $\equiv$ is the equivalence relation corresponding to $\pi$ computed by $\mathcal{E}(s,s')=\exists b\colon \mathcal{P}(s,b) \wedge \mathcal{P}(s',b)$,
the expression $\mathcal{T}(s,s') \wedge \exists b\colon 
\mathcal{P}(s,b) \wedge \mathcal{P}(s',b)$ must be computed.
%
%
%
We compute $\rightarrow\!\cap\!\equiv$ directly using a custom BDD algorithm.
%
%
%
%
The \texttt{inert} algorithm takes parameters $\mathcal{T}(s,s')$ ($\mathcal{T}$ may contain other variables ordered after $s,s'$) and two copies of $\mathcal{P}(s,b)$: $\mathcal{P}^s$ and $\mathcal{P}^{s'}$.
%
%
%It optionally also takes a parameter $\mathcal{A}(a)$, representing a set of action labels to restrict the result to (e.g. $\tau$ for only inert $\tau$-transitions).
%
%
The algorithm matches $\mathcal{T}$ and $\mathcal{P}^s$ on valuations of variables $s$, and $\mathcal{T}$ and $\mathcal{P}^{s'}$ on valuations of variables $s'$.
%
%
See Algorithm~\ref{alg:inert}, and also Figure~\ref{fig:inert} for a schematic overview.
%
%
When in the recursive call all valuations to $s$ and $s'$ have been matched, with $S_s,S_{s'}\subseteq S$ the sets of states represented by these valuations, then $\mathcal{T}$ is the set of actions that label the transitions between states in $S_s$ and $S_{s'}$, $\mathcal{P}^s$ is the block that contains all $S_s$ and $\mathcal{P}^{s'}$ is the block that contains all $S_{s'}$.
Then if $\mathcal{P}^s\neq\mathcal{P}^{s'}$, the transitions are not inert and \texttt{inert} returns \texttt{False}, removing the transition from $\mathcal{T}$.
%
%
Otherwise, $\mathcal{T}$ (which may still contain other variables ordered after $s,s'$, such as action labels), is returned.


